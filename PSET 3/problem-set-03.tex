\documentclass[12pt,twoside]{article}
\usepackage{amsmath}
\usepackage{listings}

\input{../macros}
\solutionsetfalse
\gradernotesetfalse
\renewcommand{\theproblemsetnum}{3}                            

%\pagecolor{black}
%\color{white}

\begin{document}
\input{../dates}
\ifsolutionset
\pset{ps3sol}{Problem Set \theproblemsetnum~Solutions}
\renewcommand{\solution}[1]{ \bigskip
  {\bf Solution:} #1
  \bigskip
}
\else
\pset{ps3}{Problem Set \theproblemsetnum}
\fi


\medskip
\noindent

%%%%%%%%%%%%%%%%%%%%%%%%%%%%%%%%%%%%%%%%%%%%%%%%%%%%%%%%%%%
% Introduction
%%%%%%%%%%%%%%%%%%%%%%%%%%%%%%%%%%%%%%%%%%%%%%%%%%%%%%%%%%%




\noindent This problem set is due \textbf{at 8:00pm} on \textbf{Friday, December 6, 2024}.

\smallskip
\ifsolutionset
\else

\begin{itemize}
%\item This assignment, like later assignments, consists of \emph{exercises} and \emph{problems}. \textbf{Hand in solutions to the problems only.} However, you should work out the exercises for yourself, since they will help you learn the course material. You are responsible for the material they cover.

\item
The TAs will provide a detailed document describing how you should submit your PDF and code on Google Classroom (we may use gradescope for some things). Make sure you read it! We suggest that you perform a trial submission prior to the deadline to make sure that everything works for you -- you can overwrite that submission with a new one up to the deadline.


\item
We require that written solutions are submitted as a PDF file, \textbf{typeset on \LaTeX}, using the template available on Google Classroom. You must \textbf{show your work} for written solutions. Each solution should start on a new page.


%\item
%Some problems are marked \textbf{NON-COLLABORATIVE}. You must solve these problems on your own without discussing them with anyone. You are welcome to collaborate on the rest, but you \textbf{MUST} write up the solutions entirely in your own words.

\item
We will occasionally ask you to ``give an algorithm'' to solve a problem. 
Your write-up should take the form of a short essay. 
Start by defining the problem you are solving and stating what your
results are.  Then provide: (a) a description of the algorithm in English and, if helpful, pseudo-code;
(b) a proof sketch for the correctness of the algorithm; and (c)  
an analysis of the running time.
\item 
We will give full credit \textbf{only} for correct solutions that are described
clearly and convincingly. 
\end{itemize}

\fi



%%%%%%%%%%%%%%%%%%%%%%%%%%%%%%%%%%%%%%%%%%%%%%%%%%%%%%%%%%%
% Exercises
%%%%%%%%%%%%%%%%%%%%%%%%%%%%%%%%%%%%%%%%%%%%%%%%%%%%%%%%%%%

%\newpage
%\begin{center}
%{\large \bf EXERCISES (NOT TO BE TURNED IN)}
%\end{center}
%
%\medskip
%\noindent
%\textbf{Hill cipher variants}
%
%\begin{itemize}
%  \item Blah
%  \item Blah blah
%\end{itemize}
%
%\noindent
%\textbf{Transposition ciphers}
%
%\begin{itemize}
%  \item Blah
%  \item Blah
%\end{itemize}


%%%%%%%%%%%%%%%%%%%%%%%%%%%%%%%%%%%%%%%%%%%%%%%%%%%%%%%%%%%
% Problems
%%%%%%%%%%%%%%%%%%%%%%%%%%%%%%%%%%%%%%%%%%%%%%%%%%%%%%%%%%%

\newpage


\begin{problems}

\problem[Some basics]
\points{60}

\begin{problemparts}

\problempart\points{20}

An element $a \in Z_n - {0}$ is said to be a zero divisor modulo $n$ if $ab \equiv 0 \mod n$ for some $b \in Z_n - {0}$. Each of the following questions is worth 4 points.

\begin{enumerate}
\item Explain why there are no zero divisors in $Z_p$ when $p$ is prime.
\item Find a zero divisor in $Z_{39}$.
\item What is the value of the first element to repeat: $5^1, 5^2, 5^3, \dots \mod 39$?
\item Do you think it is possible to find a non-zero number $x \in Z_{39}$ and a number $k\geq 0$ such that
$x^k \equiv 0 \mod 39$? Why or why not? Would your answer change for some RSA modulus $n$ other than 39?
\item Suppose n is not required to be an RSA modulus. Can you find numbers $n,x,k$ such that $x \not\equiv 0$ but $x^k  \equiv 0 \mod n$?
\end{enumerate}

In all cases, justify your answers.

\begin{Solutions}
\begin{enumerate}
\item
\begin{enumerate}
\item For any prime $p > 2$ (not 2 as we need both $a,b \neg\equiv 0 \pmod{p}$ ), suppose there exists a zero divisor $a$ such that $a\in Z_p - 0$ and $ab \equiv 0 \pmod{p}$ for some $b \in Z_p - 0$.

In terms of integers, this means that 
\begin{math}
$a\times b = p\times q$
\end{math}
for some $q\n Z$.

Now, the prime $p$ is an integer. So for the above equality to hold, $\frac{ab}{q}$ is also an integer and is exactly equal to $p$. $\rightarrow (1)$

Now, $a,b \in Z_p-0$ so, we know they cannot be equal to $p$. 

So $q \neq a$ and $q \neq b$. $\rightarrow (2)$

Hence, on dividing the $ab$ with $q$, using (1) and (2) we can conclude that we get the product of two non-trivial integers. This would imply that $p$ is a composite number, which is a contradiction to our assertion that $p$ is a prime number.

\item The zero divisors of $39$ are: $3, 13$.

The factors of the integer $39$ are $1,3,13,39$ (we can consider the factors in the integer domain as all factors of $n$ fall within $Z_n$).

$1$ and $39$ cannot be zero divisors modulo $39$ as $39 \equiv 0 \pmod{39}$ so $39 \neg\in Z_{39} - 0$ (both have to be elements of $Z_{39} - 0.

Now, both $3, 13 \in Z_{39} \pmod{39} - 0$. And $3 \times 13 = 39 \equiv 0 \pmod{39}$, so $3\times 13 \equiv 0 \pmod{39}$ which satisfies the requirements for zero divisors.

\item The first element to repeat would be the one right after $5^x \equiv 1 \pmod{39}$ for some $x \in \{1,2,3, \dots, \}$

The first index $i$ at which $5^i$ is $1$ is $38$. So the first element to repeat is $5^{39}$.

(TODO)

\item No this is not possible. As this would then mean that $39\times q = x^k \Rightarrow 39 = \frac{x^k}{q}$ and $\frac{x^k}{q}$ is an integer not equal to $39$ and $q\neq x^k$. Hence, this would imply that $39$ is a product of primes, which is in contradiction with $39$ being a prime itself.

For some RSA modulus $n = p\times q$, since $p\in Z_{n}$ and $p^k = $

(TODO)

\end{enumerate}
\item An efficient algorithm for computing the euclidean $gcd$ of $n$ numbers is using taking the gcd of the first members of the sequence, using the output with the 3rd member of the sequence and continuing so iteratively.

The pseudocode of the algorithm is as follows:

\begin{verbatim}
gcd(a,b) = {gcd(b-a, a) if b >a}
		{gcd(a-b,b) if a > b}
		{a if a = b}
		

\end{verbatim}
\end{enumerate}
\end{Solution}

\problempart\points{10} 
The definition of greatest common divisor can be extended naturally to a sequence of numbers $(a_1,a_2,a_3,\dots,a_k)$, not all of which are zero.
Namely, it is the largest integer $d \geq 1$ such that $d | a_j \forall j=1,2,\dots,k$. Describe an efficient algorithm for computing $gcd(a_1,a_2,a_3,\dots,a_k)$, and explain why it computes the correct answer.

\problempart\points{15}

Each of the following problems is worth 5 points. You must show your work (you will get 0 otherwise).

\begin{enumerate}
\item Euler's totient function: Compute $\phi(2200)$;
\item Euler theorem: Compute $3^{591207} \mod 2200$
\item Use the extended Euclidean algorithm to solve the Diaphantine equation: $539x - 1387y = 1$
\end{enumerate}


\problempart\points{15}

Bob's public RSA key is $n = 1501, e = 323$. Eve manages to learn that his decryption key is $d = 539$. Implement the randomized factoring algorithm from your slides. Use your program to factor $n$. Once you have the factorization of $n$, compute $\phi(n)$, and check your
answer by verifying that $ed \equiv 1 \mod \phi(n)$.

\end{problemparts}



\problem[Primitive Roots]
\points{15}

\begin{problemparts}

\problempart\points{7}
Find a primitive root $g$ of $p = 761$ and use the Lucas test to prove that you have one.

\problempart\points{8}
Find a non-trivial (not 1 or $p-1$) number $g \in Z^*_{761}$ that fails to be a primitive root of $p$, and use the Lucas test to prove your answer correct.
\end{problemparts}

\problem[Secret Sharing]
\points{25}

Implement simple $k$ out of $n$ Shamir secret sharing as described in class. You don't need any network communication: just start with a secret value $s$ and whatever public values you need (primes etc.), and then demonstrate: splitting, reconstruction, addition of two shares, addition of a share with a public value, multiplication of a share with a public value. Comment your code in detail.

For 25 points of extra credit, you can implement multiplication of two shares. For a further 15 points, you can implement Beaver triples.


\end{problems}
\end{document}
